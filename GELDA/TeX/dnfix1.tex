\documentstyle[dgelda,a4wide]{article}

\begin{document}

\begin{center}
  {\bf DNFIX1 --- Library Routine Document (Version 1.1)}
\end{center}

\section{Purpose}

DNFPEN is part of the code DGELDA \cite{KunMRW95} which solves linear
DAEs with variable coefficients of the form
\begin{eqnarray*}
  E(t)\dot{x}(t) &=& A(t)x(t) + f(t)\\
  x(t_0) &=& x_0.
\end{eqnarray*}

DNFIX1 performs the following steps:

\begin{enumerate}
\item Compute the strangeness index $\mu$ and the other characteristic
  values of the matrix pair $(E(t), A(t))$ and time T.
\item Compute the matricies $EQ, AQ$ und the vector $FQ$ of the extended
  derivative array of stage $\mu$ at time T.
\item Check if the strangeness index and the other characteristic values
  have changed.
\item Compute the coefficients of an equivalent starngeness free DAE
  at time T. The two DAEs are equivalent in the sense that their
  solutions are identical.
\end{enumerate}

\section{Specification}

\begin{verbatim}
      SUBROUTINE DNFIX1(EDIF, ADIF, FDIF, N, T, LMAX, M, ID, IA, IU,
     $                  IREQ, MQ, IDQ, IAQ, IUQ, E, LDE, A, LDA, F,
     $                  EQ, LDEQ, AQ, LDAQ, FQ, Z1, LDZ1, Z2Q, LDZ2Q,
     $                  AH,LDAH, IPAR, RPAR, RWORK, LRWORK, MCONST,
     $                  IERR)
      EXTERNAL         EDIF, ADIF, FDIF
      DOUBLE PRECISION T
      INTEGER          IERR, ID, IA, IU, IDQ, IAQ, IUQ, IREQ,
     $                 LDA, LDAH, LDAQ, LDE, LDEQ, LDZ1, LDZ2Q,
     $                 LMAX, LRWORK, M, MQ, N
      LOGICAL          MCONST
      DOUBLE PRECISION A(LDA,*), AH(LDAH,*),
     $                 AQ(LDAQ,*), E(LDE,*), EQ(LDEQ,*),
     $                 F(*), FQ(*), RPAR(*), RWORK(*),
     $                 Z1(LDZ1,*), Z2Q(LDZ2Q,*)
      INTEGER          IPAR(*)
\end{verbatim}

\section{Argument List}

\subsection{User--supplied Subroutines}

\begin{entry}{EDIF}{User--supplied function}
  This is a subroutine which the user provides to define the matrix $E(t)$
  and its derivatives. It has the form
  \begin{center}
    SUBROUTINE EDIF(N,T,IDIF,E,LDE,IPAR,RPAR,IERR).
  \end{center}
  The subroutine takes as input the number of equations N, the time T
  and the integer parameter IDIF. EDIF must not alter these input
  parameters or the leading dimension LDE of the array E. As output,
  the subroutine produces the IDIF-th derivative of $E(t)$ at time T
  in the leading N by N part of the array E. The integer flag IERR is
  always zero on input and EDIF should alter IERR only if IDIF is
  larger than the highest derivative of $E(t)$ the subroutine provides
  (set $\mbox{IERR}=-2$) or if another problem occurs (set
  $\mbox{IERR}=-1$). IPAR and RPAR are integer and real arrays which
  can be used for the communication between the user's calling program
  and the subroutines EDIF, ADIF and FDIF.

  In the calling program, EDIF must be declared as external.
\end{entry}
\begin{entry}{ADIF}{User--supplied function}
  This is a subroutine which the user provides to define the matrix $A(t)$
  and its derivatives. It is of the form
  \begin{center}
    SUBROUTINE ADIF(N,T,IDIF,A,LDA,IPAR,RPAR,IERR)
  \end{center}
  and the input and output parameters are similar to these of EDIF.
\end{entry}

\begin{entry}{FDIF}{User--supplied function}
  This is a subroutine which the user provides to define the vector $f(t)$
  and its derivatives. It is of the form
  \begin{center}
    SUBROUTINE FDIF(N,T,IDIF,F,IPAR,RPAR,IERR)
  \end{center}
  and the input and output parameters are similar to these of
  EDIF. Except that the first N elements of the 1-dimensional array F
  contain the IDIF-th derivative of $f(t)$ at time T. Note further,
  since F is a 1-dimensional array no leading dimension is needed.
\end{entry}

\subsection{Arguments In}

\begin{entry}{N}{INTEGER}
  The number of equations in the DAE system. \\
  N $\ge$ 1.
\end{entry}

\begin{entry}{T}{DOUBLE PRECISION}
  The time T.
\end{entry}

\begin{entry}{LMAX}{INTEGER}
  The maximal strangeness index.\\
  LMAX $\ge$ 1.
\end{entry}

\begin{entry}{M}{INTEGER}
  The strangeness index $\mu$ of the DAE.
\end{entry}

\begin{entry}{ID}{INTEGER}
  The number $d_{\mu}$ of differential components at the previous step.
\end{entry}

\begin{entry}{IA}{INTEGER}
  The number $a_{\mu}$ of algebraic components at the previous step.
\end{entry}

\begin{entry}{IU}{INTEGER}
  The number $u_{\mu}$ of undetermined components at the previous step.
\end{entry}

\begin{entry}{IREQ}{INTEGER}
  The rank of $EQ$ at the previous step .
\end{entry}

\begin{entry}{LDE}{INTEGER}
  The leading dimension of array E as declared in the calling program.\\
  LDEQ $\ge $ N.
\end{entry}

\begin{entry}{LDA}{INTEGER}
  The leading dimension of array A as declared in the calling program.\\
  LDA $\ge$ N.
\end{entry}

\begin{entry}{LDEQ}{INTEGER}
  The leading dimension of array EQ as declared in the calling
  program. \\
  LDEQ $\ge$ (LMAX+1) N.
\end{entry}

\begin{entry}{LDAQ}{INTEGER}
  The leading dimension of array AQ as declared in the calling
  program.\\
  LDAQ $\ge$ (LMAX+1) N.
\end{entry}

\begin{entry}{LDZ1}{INTEGER}
  The leading dimension of array Z1 as declared in the calling
  program.\\
  LDZ1 $\ge$ N.
\end{entry}

\begin{entry}{LDZ2Q}{INTEGER}
  The leading dimension of array Z2Q as declared in the calling
  program.\\
  LDZ2Q $\ge$ (LMAX+1) N.
\end{entry}

\begin{entry}{LDAH}{INTEGER}
  The leading dimension of array AH as declared in the calling
  program.\\
  LDAH $\ge$ (LMAX+1) N.
\end{entry}

\subsection{Arguments Out}

\begin{entry}{IREQ}{INTEGER}
  The rank of $EQ$ at time T.
\end{entry}

\begin{entry}{MQ}{INTEGER}
  The strangeness index $\mu$ of the DAE at time T.
\end{entry}
\begin{entry}{IDQ}{INTEGER}
  The number $d_{\mu}$ of differential components at time T.
\end{entry}

\begin{entry}{IAQ}{INTEGER}
  The number $a_{\mu}$ of algebraic  components at time T.
\end{entry}

\begin{entry}{IUQ}{INTEGER}
  The number $u_{\mu}$ of undetermined components at time T.
\end{entry}

\begin{entry}{E}{DOUBLE PRECISION array of DIMENSION (LDE,*)}
  The leading N by N part of this array contains the matrix
  $E$ of the strangeness free DAE at time T.
\end{entry}

\begin{entry}{A}{DOUBLE PRECISION array of DIMENSION (LDA,*)}
  The leading N by N part of this array contains the matrix
  $A$ of the strangeness free DAE at time T.
\end{entry}

\begin{entry}{F}{DOUBLE PRECISION array of DIMENSION (*)}
  The leading N elemnts of this array contain the vector $F$ of the
  strangeness free DAE at time T.
\end{entry}

\begin{entry}{EQ}{DOUBLE PRECISION array of DIMENSION (LDEQ,*)}
  The leading N(MD+1) by N(MD+1) part of this array contains the
  matrix $EQ$ of stage MD at time T.
\end{entry}

\begin{entry}{AQ}{DOUBLE PRECISION array of DIMENSION (LDAQ,*)}
  The leading N(MD+1) by N(MD+1) part of this array contains the
  matrix $AQ$ of stage MD at time T.
\end{entry}

\begin{entry}{FQ}{DOUBLE PRECISION array of DIMENSION (*)}
  The leading N(MD+1) elemnts of this array contain vector contain the
  vector $FQ$ of stage MQ at time T.
\end{entry}

\begin{entry}{Z1}{DOUBLE PRECISION array of DIMENSION (LDZ1,*)}
  The leading N by N part of this array contains the transformation
  matrix Z1 which extracts the differential part.
\end{entry}

\begin{entry}{Z2Q}{DOUBLE PRECISION array of DIMENSION (LDZ2Q,*)}
  The leading N(MD+1) by IAQ part of this array contains the
  transformation matrix Z2Q which extracts the algebraic part.
\end{entry}

\begin{entry}{AH}{DOUBLE PRECISION array of DIMENSION (LDAH,*)}
  This array is used as workarray.
\end{entry}

\begin{entry}{IPAR}{INTEGER array of DIMENSION (*)}
  This integer array can be used for communication between the calling
  programm and the EDIF, ADIF and FDIF subroutines.
\end{entry}

\begin{entry}{RPAR}{DOUBLE PRECISION array of DIMENSION (*)}
  This real array can be used for communication between the calling
  programm and the EDIF, ADIF and FDIF subroutines.

  IPAR and RPAR are not altered by DNFIND. If these arrays are used,
  they must be dimensioned in the calling program and in EDIF, ADIF
  and FDIF as arrays of appropriate length. Otherwise, ignore these
  arrays by treating them as dummy arrays of length one.
\end{entry}

\subsection{Workspace}
\noindent
{\bf RWORK -- {DOUBLE PRECISION array of DIMENSION at least (LRWORK)}.}
\medskip

\begin{entry}{LRWORK}{INTEGER}
  The the length RWORK.

  LRWORK $\ge 3 \mbox{((LMAX+1)N)}^2 + 6 \mbox{(LMAX+1)N}$.

  For good performance, LRWORK should generally be larger.
\end{entry}

\subsection{Tolerances}
None.

\subsection{Mode Parameters}

\begin{entry}{MCONST}{LOGICAL}
  Indicates whether the matricies $E(t)$ and $A(t)$ or constant as
  follows:
  \begin{description}
  \item[MCONST=.TRUE.] $E(t)$ and $A(t)$ are both constant matricies;
  \item[MCONST=.FLASE] $E(t)$ or $A(t)$ is a time dependent matrix.
  \end{description}
\end{entry}

\subsection{Warning Indicator}
None.

\subsection{Error Indicator}

\begin{entry}{IERR}{INTEGER}
  Unless the routine detects an error (see next section), IERR
  contains $0$ on exit.
\end{entry}

\section{Warnings and Errors detected by the Routine}

\ierr{-1}{An error occured in EDIF, ADIF or FDIF.}
\ierr{-2}{The input parameter IDIF of EDIF, ADIF or FDIF was
  larger then the highest derivative the routine provides.}
\ierr{-22}{Failed to determine strangeness index.}
\ierr{-23}{Failed to compute an equivalent strangeness index $0$
  system.}
\ierr{-26}{Unable to continue due to a change in characteristic
  values.}

\section{Method}
If we denote the
$i$-th derivative by a $\mbox{superscript}^{(i)}$ then EQ, AQ and
FQ of stage $L$ are defined componentwise by

\begin{eqnarray*}
  (EQ)_{ij} &:=& {i \choose j} E^{(i-j)}(t)
  - {i \choose j+1}A^{(i-j-1)}(t),\ i,j=0,\dots,L \\
  (AQ)_{ij} &:=& \left\{
    \begin{array}{cl}
      A^{(i)}(t) & \mbox{for } i=0,\dots,L,\ j=0 \\
      0          & \mbox{otherwise}
    \end{array}
  \right. \\
  (FQ)_i &:=& f^{(i)}(t),\ i=0,\dots,L
\end{eqnarray*}

The characteristic values of the matrix pair (EQ,AQ) of stage $L$ are
defined by

\begin{eqnarray*}
  r_L &=& \rank EQ \\
  a_L &=& \rank (Z^* AQ\ T) \\
  s_L &=& \rank (V^* Z^* AQ\ T') \\
  d_L &=& r_L - s_L \\
  u_L &=& N - r_L - a_L - s_L
\end{eqnarray*}
where

\begin{eqnarray*}
    T &&\mbox {\rm basis of }\kernel EQ\\
    Z &&\mbox{\rm basis of }\corange EQ=\kernel EQ^* \\
    T' &&\mbox{\rm basis of }\cokernel EQ=\range EQ^*\\
    V &&\mbox{\rm basis of }\corange (Z^*AQ\ T).
\end{eqnarray*}

Recursive formulas are used to compute the strangeness index $\mu$ and
the characteristics of the original pair $(E(t), A(t))$ at time T in
terms of the above values.

All information we need to compute the strangeness free DAE is hidden
in the extended derivative array $EQ$, $AQ$ anf $FQ$.

\section{References}

\begin{thebibliography}{1}
\bibitem{KunM92}
P.~Kunkel and V.~Mehrmann.
\newblock A new class of discretization methods for the solution of linear
  differential-algebraic equations.
\newblock Materialien LXII , FSP Mathematisierung, Universit{\"a}t Bielefeld,
  1992.
\newblock To appear in SIAM J. Numer. Anal.

\bibitem{KunMRW95}
P.~Kunkel, V.~Mehrmann, W.~Rath, and J.~Weickert.
\newblock {GELDA}: A software package for the solution of general linear
  differential algebraic equations.
\newblock Preprint SPC 95\_8, TU Chemnitz-Zwickau, February 1995.

\end{thebibliography}

%\section{Numerical Aspects}


%\section{Further Comments}

\end{document}
% Local Variables:
% mode: latex
% TeX-master: t
% End:
