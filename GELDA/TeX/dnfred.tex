\documentstyle[dgelda,a4wide]{article}

\begin{document}

\begin{center}
  {\bf DNFRED --- Library Routine Document (Version 1.1)}
\end{center}

\section{Purpose}
DNFRED is part of the code DGELDA \cite{KunMRW95} which solves linear
DAEs with variable coefficients of the form 
\begin{eqnarray*}
  E(t)\dot{x}(t) &=& A(t)x(t) + f(t)\\
  x(t_0) &=& x_0.
\end{eqnarray*}

DNFRED takes as input the coefficients of the extended derivative
array at a fixed time and computes the coefficients of a strangeness
free DAE, which is equivalent to the original DAE. The two DAEs are
equivalent in the sense that their solutions are identical. 

\section{Specification}

\begin{verbatim}
      SUBROUTINE DNFRED(N, NQ, ID, IA, IU, IREQ, EQ, LDEQ, AQ, LDAQ, FQ, 
     $                  Z2Q, LDZ2Q, Z1, LDZ1, E, LDE, A, LDA, F, AH,
     $                  LDAH, RWORK, LRWORK, IERR)
      INTEGER          IA, ID, IERR, IREQ, IU, LDA, LDAQ, LDEQ, LDE, 
     $                 LDZ1, LDZ2Q, LDAH, LRWORK, N, NQ
      DOUBLE PRECISION A(LDA,N), AH(LDAH,NQ), AQ(LDAQ,NQ), E(LDE,N),
     $                 EQ(LDEQ,NQ), F(N), FQ(NQ), RWORK(*), Z1(LDZ1,N),
     $                 Z2Q(LDZ2Q,NQ) 
\end{verbatim}

\section{Argument List}

\subsection{Arguments In}

\begin{entry}{N}{INTEGER}
  The number of equations in the DAE system.\\
  N $\ge$ 1.
\end{entry}

\begin{entry}{NQ}{INTEGER}
  The size of $EQ$ and $AQ$. I.e., NQ = ($\mu$+1)N, where $\mu$ is
  the strangeness index of the DAE. 
\end{entry}

\begin{entry}{ID}{INTEGER}
  The number $d_{\mu}$ of differential components.
\end{entry}

\begin{entry}{IA}{INTEGER}
  The number $a_{\mu}$ of algebraic components. 
\end{entry}

\begin{entry}{IU}{INTEGER}
  The number $u_{\mu}$ of undetermined components. 
\end{entry}

\begin{entry}{IREQ}{INTEGER}
  The rank of EQ.
\end{entry}

\begin{entry}{EQ}{DOUBLE PRECISION array of DIMENSION (LDEQ,*)}
  The leading NQ by NQ part of this array contains the
  matrix $EQ$.
\end{entry}

\begin{entry}{LDEQ}{INTEGER}
  The leading dimension of array EQ as declared in the calling program.

  LDEQ $\ge $ NQ.
\end{entry}

\begin{entry}{AQ}{DOUBLE PRECISION array of DIMENSION (LDAQ,*)}
  The leading NQ by NQ part of this array contains the
  matrix $AQ$.
\end{entry}

\begin{entry}{LDAQ}{INTEGER}
  The leading dimension of array AQ as declared in the calling program.

  LDAQ $\ge$ NQ.
\end{entry}

\begin{entry}{FQ}{DOUBLE PRECISION array of DIMENSION (*)}
  The leading NQ elements of this array contains the vector $FQ$.
\end{entry}

\begin{entry}{Z2Q}{DOUBLE PRECISION array of DIMENSION (LDZ2Q,*)}
  The leading NQ by NQ part of this array must contain the
  right orthogonal factor of the SVD $EQ = ZQ\ \Sigma\ V^T$.\\
  {\bf Note:} this array is overwritten.
\end{entry}

\begin{entry}{LDZ2Q}{INTEGER}
  The leading dimension of array AQ as declared in the calling program.

  LDZ2Q $\ge$ NQ.
\end{entry}

\begin{entry}{LDZ1}{INTEGER}
  The leading dimension of array Z1 as declared in the calling program.

  LDZ1 $\ge$ N.
\end{entry}

\begin{entry}{LDE}{INTEGER}
  The leading dimension of array E as declared in the calling program.

  LDE $\ge$ N.
\end{entry}

\begin{entry}{LDA}{INTEGER}
  The leading dimension of array A as declared in the calling program.

  LDA $\ge$ N.
\end{entry}

\begin{entry}{LDAH}{INTEGER}
  The leading dimension of array AH as declared in the calling program.

  LDAH $\ge$ NQ.
\end{entry}

\subsection{Arguments Out}

\begin{entry}{Z2Q}{DOUBLE PRECISION array of DIMENSION (LDZ2Q,*)}
  The leading NQ by IA part of this array contains the
  transformation matrix Z2Q which extracts the algebraic part. 
\end{entry}

\begin{entry}{Z1}{DOUBLE PRECISION array of DIMENSION (LDZ1,*)}
  The leading N by ID part of this array contains the
  transformation matrix Z1 which extracts the differential part. 
\end{entry}

\begin{entry}{E}{DOUBLE PRECISION array of DIMENSION (LDE,*)}
  The leading N by N part contains $E$ of the strangeness free DAE. 
\end{entry}

\begin{entry}{A}{DOUBLE PRECISION array of DIMENSION (LDA,*)}
  The leading N by N part contains $A$ of the strangeness free DAE. 
\end{entry}

\begin{entry}{F}{DOUBLE PRECISION array of DIMENSION (*)}
  The leading N elements contains $f$ of the strangeness free DAE. 
\end{entry}

\begin{entry}{AH}{DOUBLE PRECISION array of DIMENSION (LDAH,*)}
  This array is used as workarray.
\end{entry}

\subsection{Workspace}

\noindent
{\bf RWORK -- {DOUBLE PRECISION array of DIMENSION (see LRWORK)}.}
\medskip

\begin{entry}{LRWORK}{INTEGER}
  The dimension of workspace array RWORK. 

  LRWORK $\ge 3\mbox{N} + 2\mbox{NQ} + 2\mbox{NQ}^2$.

  For good performance, LRWORK should generally be larger.
\end{entry}

\subsection{Tolerances}
None.

\subsection{Mode Parameters}
None.

\subsection{Warning Indicator}
None.

\subsection{Error Indicator}
\begin{entry}{IERR}{INTEGER}
  Unless the routine detects an error (see next section), IERR
  contains $0$ on exit.
\end{entry}

\section{Warnings and Errors detected by the Routine}

\ierr{-3}{An argument of DGESVD had an illegal value.}
\ierr{-4}{DGESVD failed to converge.}
\ierr{-5}{LRWORK $< 3\mbox{N} + 2\mbox{NQ} + 2\mbox{NQ}^2$.}

\section{Method}

All information we need to transform the original DAE is hidden in the
extended derivative array EQ, AQ anf FQ. DNFOTR is used to compute the 
orthogonal transformation matricies Z1 and Z2Q.

\section{References}

\begin{thebibliography}{1}
\bibitem{KunM92}
P.~Kunkel and V.~Mehrmann.
\newblock A new class of discretization methods for the solution of linear
  differential-algebraic equations.
\newblock Materialien LXII , FSP Mathematisierung, Universit{\"a}t Bielefeld,
  1992.
\newblock To appear in SIAM J. Numer. Anal.

\bibitem{KunMRW95}
P.~Kunkel, V.~Mehrmann, W.~Rath, and J.~Weickert.
\newblock {GELDA}: A software package for the solution of general linear
  differential algebraic equations.
\newblock Preprint SPC 95\_8, TU Chemnitz-Zwickau, February 1995.

\end{thebibliography}

%\section{Numerical Aspects}

%\section{Further Comments}

\end{document}
% Local Variables: 
% mode: latex
% TeX-master: t
% End: 





