\documentstyle[dgelda,a4wide]{article}

\begin{document}

\begin{center}
  {\bf DNFVAL ---  Library Routine Document (Version 1.1)}
\end{center}

\section{Purpose}

This code is part of the code DGELDA \cite{KunMRW95 }which solves
linear DAEs with variable coefficients of the form 

\begin{eqnarray*}
  E(t)\dot{x}(t) &=& A(t)x(t) + f(t)\\
  x(t_0) &=& x_0.
\end{eqnarray*}

DNFVAL computes the characteristic values of a matrix pair $(EQ, AQ)$
and the right orthogonal factor ZQ of the SVD $EQ = ZQ\ \Sigma\ V^T$.

\section{Specification}

\begin{verbatim}
      SUBROUTINE DNFVAL(N, NQ, COND, EQ, LDEQ, AQ, LDAQ, R0, A0, S0, D0,
     $                  U0, ZQ, LDZQ, RWORK, LRWORK, IERR)
      DOUBLE PRECISION COND
      INTEGER          A0, D0, IERR, LDE, LDA, LDZQ, LRWORK, N, 
     $                 R0, S0, U0 
      DOUBLE PRECISION A(LDA,*), E(LDE,*), RWORK(*), ZQ(LDZQ,*)
\end{verbatim}

\section{Argument List}

\subsection{Arguments In}

\begin{entry}{N}{INTEGER}
  The number of equations of the original DAE.\\
  N $\ge$ 1.
\end{entry}

\begin{entry}{NQ}{INTEGER}
  The size of the square matricies $EQ$ and $AQ$.\\
  NQ $\ge$ N.
\end{entry}

\begin{entry}{EQ}{DOUBLE PRECISION array of DIMENSION (LDEQ,*)}
  The leading NQ by NQ part of this array must contain the matrix $EQ$.
\end{entry}

\begin{entry}{LDEQ}{INTEGER}
  The leading dimension of array EQ as declared in the calling
  program.

  LDEQ $\ge$ NQ.
\end{entry}

\begin{entry}{AQ}{DOUBLE PRECISION array of DIMENSION (LDAQ,*)}
  The leading NQ by NQ part of this array must contain the matrix $AQ$.
\end{entry}

\begin{entry}{LDAQ}{INTEGER}
  The leading dimension of array AQ as declared in the calling
  program. 

  LDAQ $\ge$ NQ.
\end{entry}

\begin{entry}{LDZQ}{INTEGER}
  The leading dimension of array ZQ as declared in the calling
  program. 

  LDZQ $\ge$ NQ.
\end{entry}

\subsection{Arguments Out}

\begin{entry}{R0}{INTEGER}
  The rank of $EQ$.
\end{entry}

\begin{entry}{A0}{INTEGER.}
  The size of the algebraic part.
\end{entry}

\begin{entry}{S0}{INTEGER}
  The size of the strangeness part.
\end{entry}

\begin{entry}{D0}{INTEGER}
  The size of the differential part.
\end{entry}

\begin{entry}{U0}{INTEGER}
  The size of the undetermind part.
\end{entry}

\begin{entry}{ZQ}{DOUBLE PRECISION array of DIMENSION (LDZQ,*)}
  The leading NQ by NQ part of this array contains the right
  orthogonal factor of the SVD $EQ = ZQ\ \Sigma\ V^T$. 
\end{entry}

\subsection{Workspace}

\noindent
{\bf RWORK -- {DOUBLE PRECISION array of DIMENSION at least (LRWORK)}.}
\medskip

\begin{entry}{LRWORK}{INTEGER}
  The length of RWORK. 

  LRWORK $\ge 3 \mbox{NQ}^2 + 6 \mbox{NQ}$.

  For good performance, LRWORK should generally be larger.
\end{entry}

\subsection{Tolerances}
None.

\subsection{Mode Parameters}
None.

\subsection{Warning Indicator}
None.

\subsection{Error Indicator}

\begin{entry}{IERR}{INTEGER.}
  Unless the routine detects an error (see next section), IERR
  contains 0 on exit. 
\end{entry}

\section{Warnings and Errors detected by the Routine}

\ierr{-3}{An argument of DGESVD had an illegal value.}
\ierr{-4}{DGESVD failed to converge.}
\ierr{-5}{LRWORK $< 3 \mbox{NQ}^2 + 6 \mbox{NQ}$.}

\section{Method}

The characteristic values of a NQ by NQ matrix pair $(EQ, AQ)$ are
defined by 

\begin{eqnarray*}
  R0 &=& \rank EQ \\
  A0 &=& \rank (Z^* AQ T) \\
  S0 &=& \rank (V^* Z^* AQ T') \\
  D0 &=& R0 - S0 \\
  U0 &=& N - R0 - A0 - S0
\end{eqnarray*}
where 

\begin{eqnarray*}
    T &&\mbox {\rm basis of }\kernel EQ\\
    Z &&\mbox{\rm basis of }\corange EQ=\kernel EQ^* \\
    T' &&\mbox{\rm basis of }\cokernel EQ=\range EQ^*\\
    V &&\mbox{\rm basis of }\corange (Z^*AQ\ T).
\end{eqnarray*}

The characteristic values R0, A0, S0, D0, U0 are computed via three rank
decisions, which are obtained by means of singular value
decompositions (SVDs). The LAPACK subroutine DGESVD is used to compute
these SVDs.  

\section{References}

\begin{thebibliography}{1}

\bibitem{KunM92}
P.~Kunkel and V.~Mehrmann.
\newblock A new class of discretization methods for the solution of linear
  differential-algebraic equations.
\newblock Materialien LXII , FSP Mathematisierung, Universit{\"a}t Bielefeld,
  1992.
\newblock To appear in SIAM J. Numer. Anal.

\bibitem{KunMRW95}
P.~Kunkel, V.~Mehrmann, W.~Rath, and J.~Weickert.
\newblock {GELDA}: A software package for the solution of general linear
  differential algebraic equations.
\newblock Preprint SPC 95\_8, TU Chemnitz-Zwickau, February 1995.

\end{thebibliography}

%\section{Numerical Aspects}

%\section{Further Comments}

\end{document}
% Local Variables: 
% mode: latex
% TeX-master: t
% End: 





